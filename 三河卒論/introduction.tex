% !TEX root = MasterPaper.tex
\chapter{序論}
\thispagestyle{fancy}
\lhead{}
\chead{}
\rhead{}
\lfoot{} 
\cfoot{\thepage}  
\rfoot{}
%目次のページ番号をアラビア数字に指定
\setcounter{page}{1}
\pagenumbering{arabic}

近年世界で流行している新型コロナウイルス(COVID-19)の影響により,現地でのスポーツ観戦が困難になっている.
それに伴って,現地での観戦時特有の「臨場感」が失われてしまっていることが大きな問題となっている.

この問題を受け,近年ではKDDIと横浜DeNAベイスターズが実施するバーチャルハマスタなどの,次世代型のスポーツ観戦方式が登場している\cite{hamasuta}.
バーチャルハマスタとは,バーチャル空間にスタジアムを構築し,自宅からスマートフォンやパソコン,VRデバイスを使って観戦体験ができるプロジェクトである.
VR空間で,ユーザはオリジナルのアバターを用いて多くのファンと一緒に観戦し,コミュニケーションをとりながら球場の雰囲気を楽しむことができる.

しかし,この観戦方式は,観戦時にアバターが棒立ちになってしまうことや,感情表出方法が絵文字による表出のみであることから,現地でのスポーツ観戦を再現しきれておらず,臨場感を演出できていないと考えられる.
ここで,VR空間における棒立ちのアバターを感情表出を行うロボットに置き換えることで,現地でのスポーツ観戦を再現できる可能性がある.

一方で近年,自動運転を搭載した自動車が発表されるなど,人工知能を搭載した製品の普及は目覚ましいものとなっている.
使用者の暮らしに応じた機能を提示する人工知能搭載型の様々な電化製品が開発され,人工知能を搭載したロボットが囲碁でプロの棋士に勝利するなど,人工知能は人々にとって身近な存在になりつつある\cite{healsio}\cite{go}.

また,近年ではロボットに人工知能を搭載し,人とコミュニケーションをとるロボットが数多く登場している.
このようなロボットはコミュニケーションロボットと呼ばれており,その代表的なものにパーソナルロボットやペットロボットが挙げられる\cite{toyota}\cite{aibo}.
これらのロボットは人の心的状況を読み取り,感情的なコミュニケーションをとることが求められる.
さらに今後,公共の場で人の代わりにコミュニケーションをとるロボットが普及することが示唆されている\cite{deep}.

このようなコミュニケーションロボットは人と円滑なコミュニケーションを行うため,非言語情報による感情表現が重要だと言われている.
現在でも,顔表情で感情表出を行うロボットや,指と手によるジェスチャーとして手話を取り上げ,ロボットに実装した例が報告されている
\cite{kao}\cite{syuwa}.
コミュニケーションにおいて,人間に心的作用をもたらす要因は主として非言語情報である\cite{higengo}.
非言語コミュニケーションの重要性は,人間同士だけでなく,人間とロボットの場合においても同様であると考えられる.
したがって,人間と共存するコミュニケーションロボットには,非言語機能が必要になると考えられる.
そこで本研究では,非言語情報によって感情表出を行うロボット集団との試合観戦は,臨場感を演出できるか検証する.

本論文は5章で構成される.

第2章では,臨場感についての関連研究,スポーツ観戦における臨場感及び感性とロボットの関係性について述べる.
初めに,臨場感の定義と臨場感を感じる事象である「興奮」や「面白さ」の高まりに関する先行研究について述べる.
次に,スポーツ観戦における臨場感の演出に必要な要素について述べる.
最後に,ロボットに感性を導入する意義,ロボットと人間とのインタラクションについて述べる.

第3章では,スポーツ観戦における臨場感演出の定義,本研究で提案する内部モデルについての詳細及び実験環境,及び実験におけるロボット集団の動作について述べる.
本研究ではロボットが試合観戦中に感情表出をしているように見せる内部モデルを提案する.
また,本実験で用いたロボット集団とインタラクションをとる仮想環境と,仮想環境でのロボットの動作について述べる.

第4章では,提案ロボット集団との試合観戦が臨場感を演出できるか検証するための実験について述べる.
評価実験では,被験者は提案ロボット集団と比較ロボット集団の2集団と試合観戦を行い,アンケートによって試合観戦における臨場感,及びロボット集団への評価を行う.

第5章は結論であり,本研究により得られた成果についてまとめると共に,今後に残された課題について述べる.

