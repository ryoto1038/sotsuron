% !TEX root = MasterPaper.tex
\chapter{結論}
\thispagestyle{fancy}
\lhead{}
\chead{}
\rhead{}
\lfoot{} 
\cfoot{\thepage}
\rfoot{}

本研究では,スポーツ観戦における臨場感演出のためのロボット集団の振る舞いについて調査した.
先行研究では,人と顔表情で感性表現をするロボットとのコミュニケーションによって,ユーザに親しみやすい印象を与えていた.しかし,先行研究のシチュエーションでは,ロボット集団とのコミュニケーションは考えられていない.また,臨場感に関する先行研究,「興奮」や「面白さ」は臨場感を高めるための重要な要素であることが明らかになっていた.しかし,先行研究ではロボットとのインタラクションで臨場感を感じるかどうかは検証されていない.

そのため,本研究では,感情表出するロボット集団とスポーツ観戦を行うことで,ロボット集団から人への感情伝播が起こるかを検証した.また,ロボット集団との観戦によって起こる没入感や一体感,感情伝播によって臨場感を演出することは可能かを検証した.

第2章では,まず臨場感の定義についての先行研究,臨場感を感じる事象の分析についての先行研究を述べた.
次に,スポーツ観戦において臨場感を演出するための要素について述べた.
続いて,ロボットの進化についてと,ロボットへ感性を導入する意義について述べた.
最後に,ロボット集団と人間が円滑にインタラクションを行うためのロボット集団の設計について述べた.

第3章では,本研究で提案するロボット集団の内部モデル,本実験で用いた実験環境とロボット集団の動作について述べた.まず初めに,ロボット集団が実際に観戦を行っているように演出するため,勝利確率を読み取って感情生成を行うモデルについて述べた.次に,本実験で用いた環境について述べた.本実験では,ロボット集団との観戦で臨場感を演出するために,スポーツバーを模した空間を制作した.最後に,本実験で用いたロボットの動作について述べた.

第4章では,実ユーザによる評価実験について述べた.まず初めに,観戦する試合映像と,ロボット集団のパラメータについて述べた.次に,感情表出をするロボット集団との野球観戦でユーザが臨場感を感じるのかを検証した実験について述べた.被験者は提案ロボット集団と比較ロボット集団の両方と野球観戦を行い,その後アンケートを用いてそれぞれのロボット集団への印象評価と比較を行った.最後に,アンケートの結果を用いて提案ロボット集団の有効性について検討した.

本実験の結果から,提案ロボット集団との観戦で,ロボット集団からユーザに感情伝播が起こり,それによって起こる没入感や一体感によって,臨場感を感じることが分かった.
今後の課題としては,ロボット集団内でのロボット同士の感情伝播や,ロボットの表出感情の改良による更なる臨場感の演出が挙げられる.
